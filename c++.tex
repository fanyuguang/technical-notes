\documentclass[11pt,a4paper]{article}

\usepackage[space]{ctex}
\usepackage{amsmath}
\usepackage{graphicx}
\usepackage{float} 
\usepackage{subfigure}
\usepackage{listings}
\lstset{
    breaklines,
    % numbers=left, 
    % numberstyle=\tiny,
    frame=single,
    basicstyle=\ttfamily,
    language=c++,
}

\title{C++}
\author{Yuguang Fan}
\date{\today}
\begin{document}
\maketitle

\section{基本数据类型}

aaa

\section{表达式}

aaa

\section{语句}

aaa

\section{函数}

aaa

\section{类}

aaa

\section{容器}

aaa

\section{IO操作}

\subsection{文件操作}

\begin{lstlisting}[language={C++}]
#include <fstream>
\end{lstlisting}

打开或创建文本

\begin{lstlisting}[language={C++}]
string file_path = "../data/filename.txt";
fstream text_file;
text_file.open(file_path, ios::out); //打开文件并清空内容
text_file.open(file_path, ios::out | ios::app);//打开文件,并保留文件原有内容,定位到文件末尾
\end{lstlisting}

\begin{itemize}
    \item ios::in读
    \item ios::out写
    \item ios::app每次写操作前,定位到文件末尾
    \item ios::ate打开文件时,定位到文件末尾
    \item ios::trunc清空内容
\end{itemize}

打开或创建带序号文本

\begin{lstlisting}[language={C++}]
int i = 1;
char file_path[50];
sprintf(file_path, "../data/file%d.txt", i);
fstream text_file;
text_file.open(file_path, ios::out);
\end{lstlisting}

遍历读取文本信息

方法一

按词读取

\begin{lstlisting}[language={C++}]
string str;
while (text_file >> str)
{
    cout << str << endl;
}
\end{lstlisting}

按行读取

\begin{lstlisting}[language={C++}]
string str;
while(getline(text_file, str))
{
    cout << str << endl;
}
\end{lstlisting}

方法二

按词读取

\begin{lstlisting}[language={C++}]
string str;
while (!text_file.eof())
{
    text_file >> str;
}
\end{lstlisting}

按行读取

\begin{lstlisting}[language={C++}]
string str;
while (!text_file.eof())
{
    getline(text_file, str);
    cout << str;
}
\end{lstlisting}

向文本写入信息

\begin{lstlisting}[language={C++}]
string str1 = “hello”;
string str2 = “world”;
text_file << str1 << “ ”<< str2 << endl;
\end{lstlisting}

关闭文本

\begin{lstlisting}[language={C++}]
text_file.close();
\end{lstlisting}

\section{异常处理}

aaa

\section{日志}

aaa

\section{内存管理}

aaa

\section{并发}

aaa

\section{泛型}

aaa

\section{重载}

aaa

\section{模板}

aaa

\section{编程规范}

子程序一般情况下代码长度不宜超过200行

\subsection{命名规则}

\begin{itemize}
    \item 文件:全部小写,中间以下划线连接file_name.cpp
    \item 函数:每个单词首字母大写FunctionName ()
    \item 类:首字母大写ClassName
    \item 枚举类型:首字母大写EnumeratorTypes
    \item 局部变量:全部小写,以下划线连接local_variable_name
    \item 类的成员变量:全部小写,以下划线连接,以下划线结尾 class_variable_name_
    \item 结构体的成员变量:全部小写,以下划线连接struct_variable_name
    \item 全局变量:g_做前缀,全部小写,以下划线连接g_global_variable_name
    \item 常量:k做前缀,首字母大写kConstantName
    \item 宏:全部大写,以下划线连接MACRO_TYPE
    \item 名字空间:全部小写,以下划线连接namespace_names
\end{itemize}

\subsection{#include头文件引用顺序}

\begin{enumerate}
    \item 自身.cpp文件的.h文件
    \item 系统.h文件
    \item 本项目内其他自定义的.h文件
\end{enumerate}

\section{随记}

\subsection{打印时间}

\begin{lstlisting}[language={C++}]
auto now = std::chrono::system_clock::now();
std::time_t t = std::chrono::system_clock::to_time_t(now);
std::cout << std::put_time(std::localtime(&t), "%Y-%m-%d %H:%M:%S") << " ";
\end{lstlisting}

\begin{thebibliography}{99}
\bibitem{1} 参考文献1
\bibitem{2} 参考文献2
\end{thebibliography}
\end{document}
